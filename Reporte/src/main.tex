%! Author = ivanreyes
%! Date = 07/10/21

%Numero de paginas en dvi =
\documentclass[spanish]{rmf-d}

\usepackage{nopageno, rmfbib, multicol, times, epsf, amsmath, amssymb, cite}
\usepackage[utf8]{inputenc}
\usepackage[spanish]{babel}
\usepackage[]{caption2}
\usepackage{graphics}

%
\spanishdecimal{.}

\def\rmfcornisa{EDUCATION OF PHYSICS \hfill\rmf\ E \textbf{71} (2022) 10--17
\hfill JANUARY-JUNE 2022}

\newcommand{\ssc}{\scriptscriptstyle}

%
\def\rmfcintilla{{\textit Rev.\ Mex.\ Fis.\/ } {E \textbf{71} (2022) 10--17}}
\clearpage \rmfcaptionstyle \pagestyle{myheadings}
\setcounter{page}{1}

\markboth{Iván Reyes-Hernández}
{Análisis de los datos del efecto fotoeléctrico}

\begin{document}

\title{Análisis de los datos del efecto fotoeléctrico
\vspace{-6pt}}

\author{Iván Reyes-Hernández}

\address{Facultad de Ciencias, Universidad Nacional Autónoma de México,\\
Ciudad de México, México. \\
e-mail: ivanreyes@ciencias.unam.mx}

\maketitle

\recibido{10 Oct 2021}{10 Dic 2021
\vspace{-12pt}}

% Resumen español
\begin{resumen}

\vspace{1em} FAVOR DE PROPORCIONAR RESUMEN EN ESPAÑOL\vspace{1em}

\end{resumen}

\descript{  \textbf{\textit{Análisis de datos; efecto fotoeléctrico}} \vspace{0pt}}

% Resumen inglés
\begin{abstract}

\vspace{1em} FAVOR DE PROPORCIONAR RESUMEN EN INGLES\vspace{1em}

\end{abstract}

\keys{ \textbf{\textit{Data analysis; photoelectric effect}} \vspace{-8pt}}
\pacs{ \textbf{\textit{FAVOR DE PROPORCIONAR}} \vspace{-4pt}}

\begin{multicols}{2}

AGREGAR  TEXTO DEL ARTICULO


\end{multicols}
\medline

\begin{multicols}{2}
\begin{thebibliography}{99}
%\bibitem{c1}

%1
\bibitem{c1} COLOCAR AQUI LA BIBLIOGRAFIA


\end{thebibliography}

\end{multicols}

\end{document}

%
%Notas:
%
%\tabletopline\vspace{2pt}\lilahf{\sc Table I.\ {\rm Table caption}}
%\begin{center}
%\small{\renewcommand{\arraystretch}{1.3}
%\renewcommand{\tabcolsep}{1.35pc}
%\begin{tabular}{cccc}
%\hline
%\hline
%\end{tabular}}
%\end{center}
%